\documentclass[11pt,a4paper]{article}
\usepackage{acl2017}
\usepackage{times}
% \usepackage{multirow}
\usepackage{url}
\usepackage{latexsym}
\usepackage{graphicx}
\usepackage{color}
\usepackage{booktabs}
\usepackage{amsmath}
\usepackage[utf8]{vietnam}

\aclfinalcopy % Uncomment this line for the final submission

%\setlength\titlebox{5cm}

\title{Báo cáo kỹ thuật\\ Xây dựng hệ thống tách từ tiếng Việt \\ underthesea v1.1.12}

\author{
Vu Anh\\
underthesea\\
anhv.ict91@gmail.com
}
\date{}

\begin{document}
\maketitle
\begin{abstract}
Trong báo cáo này, chúng tôi mô tả chương trình tách từ tiếng Việt, được tích hợp trong phiên bản underthesea phiên bản 1.1.12.
Mã nguồn của chương trình được open source tại \href{https://github.com/undertheseanlp/word_tokenize}{github}.

\end{abstract}

\section{Giới thiệu}

Tách từ là một bài toán quan trọng trong việc xử lý rất nhiều ngôn ngữ. Đối với tiếng Việt, nhiệm vụ này khá khó khăn do một từ tiếng Việt thường gồm nhiều tiếng ghép lại. Ví dụ như từ \textit{giáo viên} gồm hai tiếng \textit{giáo} và \textit{viên}.

\section{Các công trình liên quan}

Bài toán tách từ tiếng Việt đã được nghiên cứu từ khá lâu.

\section{Mô tả hệ thống}

\subsection{Hệ thống tách từ}

Hệ thống tách từ trong underthesea được chia làm hai bước. Bước đầu tiên là bước tiền xử lý. Trong bước này, văn bản được tách câu và tokenize sử dụng regular expression. Bước thứ hai, các từ được biểu diễn dưới dạng một bài toán gán nhãn chuỗi.

\subsection{Thuật toán Conditional Random Fields}

Thuật toán Conditional Random Fields (CRFs) ~\cite{Lafferty:2001:CRF:645530.655813} được sử dụng đã tính toán xác suất của chuỗi đầu ra cho bởi chuỗi đầu vào. Xác suất của chuỗi trạng thái $S = <s_1, s_2,..., s_T>$ cho bởi quan sát $O = <o_1, o_2, ..., o_T>$ được tính bởi công thức:

$$P(s|o) = \frac{1}{Z_o} exp( \sum_{t=1}^{T} \sum_{k} \lambda_k x f_k (s_{t-1},s_t,o,t) )$$

trong đó, $f_k (s_{t-1},s_t,o,t)$ làm một hàm đặc trưng ứng với trọng số $\lambda_k$, được học thông qua quá trình huấn luyện.

\subsection{Features}
We propose conditional random fields for this problem.

Our final features
\begin{center}
\begin{tabular}{ |c|c| }
 \hline
 features & description \\
 \hline
 T[-2], T[-1], T[0], T[1], T[2] & unigram  \\
 T[-2,-1], T[-1,0], T[0,1], T[1,2] & bigram  \\
 T[-2,0], T[-1,1], T[0,2] & trigram \\
 T[-1].isdigit, T[0].isdigit, T[1].isdigit & digit
 \hline
\end{tabular}
\end{center}

\section{Thực nghiệm}

\subsection{Data sets}

Dữ liệu VLSP 2013.

\subsection{Evaluation Measures}

We used Precision, Recall, F1 score as evaluation measures.

$$F_1 = \frac{2*P*R}{P + R}$$

where P (Precision), and R (Recall) are determined as follows:

$$P = \frac{{NE}_{true}}{NE_{sys}}$$

$$R = \frac{{NE}_{true}}{NE_{ref}}$$

where

$NE_{true}$: The number of NEs in gold data

$NE_{sys}$: The number of NEs in recognizing system

$NE_{true}$: The number of NEs which is correctly recognized by the system

\subsection{Kết quả}

We conduct our experiment in VLSP 2013 dataset, the result show we archive 97.3\%

\begin{center}
\begin{tabular}{ |c|c| }
 \hline
 system & features & result \\
 \hline
 s1 & ngram & 96.42\\
 s2 & s1 + lower & 96.45\\
 s3 & s2 + isdigit & 96.54\\
 s4 & s3 + istitle & 96.45 \\
 s5 & s4 + unigram is in dict & 96.45 \\
 s6 & s5 + bigram is in dict & 97.34 \\
 sn & full & 97.31\% \\
 \hline
\end{tabular}
\end{center}


\section{Kết luận}

Trong báo cáo này, chúng tôi đã mô tả hệ thống tách từ được tích hợp trong underthesea phiên bản 1.1.12.

\bibliography{technique_report}
\bibliographystyle{acl_natbib}

\end{document}
