%
% File eacl2017.tex
%
%% Based on the style files for ACL-2016
%% Based on the style files for ACL-2015, with some improvements
%%  taken from the NAACL-2016 style
%% Based on the style files for ACL-2014, which were, in turn,
%% Based on the style files for ACL-2013, which were, in turn,
%% Based on the style files for ACL-2012, which were, in turn,
%% based on the style files for ACL-2011, which were, in turn, 
%% based on the style files for ACL-2010, which were, in turn, 
%% based on the style files for ACL-IJCNLP-2009, which were, in turn,
%% based on the style files for EACL-2009 and IJCNLP-2008...

%% Based on the style files for EACL 2006 by 
%%e.agirre@ehu.es or Sergi.Balari@uab.es
%% and that of ACL 08 by Joakim Nivre and Noah Smith

\documentclass[11pt,a4paper]{article}
\usepackage{acl2017}
\usepackage{times}
\usepackage{multirow}
\usepackage{url}
\usepackage{latexsym}
\usepackage{graphicx}
\usepackage{color}
\usepackage{booktabs}
\usepackage{amsmath}

\aclfinalcopy % Uncomment this line for the final submission
%\def\eaclpaperid{***} %  Enter the acl Paper ID here

%\setlength\titlebox{5cm}
% You can expand the titlebox if you need extra space
% to show all the authors. Please do not make the titlebox
% smaller than 5cm (the original size); we will check this
% in the camera-ready version and ask you to change it back.

\newcommand\BibTeX{B{\sc ib}\TeX}

\title{Vietnamese Word Segmentation System \\ in underthesea v1.1.8}

\include{notation}

\author{
Vu Anh\\
underthesea\\
Hanoi, Vietnam\\
anhv.ict91@gmail.com
}
\date{}

\begin{document}
\maketitle
\begin{abstract}
In this report, we describe our word segmentation system for Vietnamese, which is integrated in underthesea version 1.1.8.
Our system is open-source and available at \url{https://github.com/undertheseanlp/word_tokenize}

\end{abstract}

\section{Introduction}

To be updated

\section{System Description}

To be updated

\section{Evaluation}

\subsection{Data sets}

To be updated

% TODO To be updated

\subsection{Evaluation Measures}

We used Precision, Recall, F1 score as evaluation measures.

$$F_1 = \frac{2*P*R}{P + R}$$

where P (Precision), and R (Recall) are determined as follows:

$$P = \frac{{NE}_{true}}{NE_{sys}}$$

$$R = \frac{{NE}_{true}}{NE_{ref}}$$

where

$NE_{true}$: The number of NEs in gold data

$NE_{sys}$: The number of NEs in recognizing system

$NE_{true}$: The number of NEs which is correctly recognized by the system

\subsection{Results}

% TODO To be updated

\section{Conclusion}

We have introduced our approach and its experimental result in word segmentation for Vietnamese text.

% TODO To be updated

\bibliography{technique_report}
\bibliographystyle{acl_natbib}

\end{document}
